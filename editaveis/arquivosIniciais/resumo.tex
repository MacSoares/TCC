\begin{resumo}

 Este trabalho vem apresentar uma nova proposta de detecção de padrões de malwares em fluxos de rede baseada em redes neurais artificiais, porém utilizando-se de conceitos de aprendizado de maquina , tais como por exemplo \textit{feature engineering} na organização dos modelos de dados a serem utilizados para o treinamento da rede. O objetivo é ao fim do projeto, possuir um sistema inteligente capaz de detectar os padrões de malware presentes em pacotes de fluxos de rede e se possível classifica-los, para que o administrador de rede possua um melhor feedback que uma simples marcação de chance de um pacote ser malicioso.

 \vspace{\onelineskip}

 \noindent
 \textbf{Palavras-chaves}: rede neural artificial. aprendizado de máquina. malware. feature engineering. fluxos de rede.
\end{resumo}
