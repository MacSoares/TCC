\begin{resumo}

 Este trabalho vem apresentar uma proposta de detecção de padrões de malwares em fluxos de rede com uso de redes neurais artificiais, modelo MLP (\textit{Multi Layer Perceptron}) e treinamento supervisionado. O objetivo é ao fim do projeto, possuir um sistema inteligente capaz de detectar os padrões de malware presentes em pacotes de fluxos de rede e se possível classifica-los, para que o administrador de rede possua um melhor feedback que uma simples marcação de chance de um pacote ser malicioso.

 \vspace{\onelineskip}

 \noindent
 \textbf{Palavras-chaves}: Multi Layer Perceptron. Backpropagation. aprendizado de máquina. malware. feature engineering. fluxos de rede.
\end{resumo}
