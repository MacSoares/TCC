\part{Metodologia}
\chapter[Metodologia]{Metodologia}

Tendo como meta a definição de um modelo de ML para ser aplicada em uma RNA a ser treinada para detecção de intrusão, a estratégia a ser seguida é o fluxo de trabalho de aprendizado de máquinas, descrito na figura 6 (\ref{fig06}).

\section{Construção do Modelo}

    O modelo de aprendizado utilizado será o de redes neurais artificiais, baseado em uma rede MLP com método de treinamento \textit{backpropagation}. Para definir o tamanho de cada camada da RNA é necessário primeiro que ocorram um pré-processamento de dados relacionado à engenharia de características, para que a partir das características escolhidas, o tamanho da camada de entrada da rede seja definido.

    Sendo assim os passos a serem seguidos para a construção do modelo inicial são:

    \begin{itemize}
        \item Pré-processamento dos dados: \\ Neste primeiro passo, é montado o modelo de dados de treinamento que será usado para apresentação e utilização dos dados. Para isso é necessário encontrar uma corelação entre os dados de entrada e o dado a ser atingido pelo processo de aprendizado. A partir do momento que a corelação é encontrada, os dados devem ser dispostos de modo que facilite sua visualização e uso, sendo assim de acordo com \cite{brink2015}, é recomendado que se convertam os dados categóricos (normamente textuais) em dados numéricos, para facilitar seu uso. Nesta faze se aplica o processo \textit{Linear Forward Selection}.

        \item Definição das camadas da rede: \\ Tendo como base a quanidade de características extraídas pelo modelo de seleção, a camada de entrada da rede possuirá um neurônio para cada característica de entrada escolhida. A camada de saída possuirá apenas um neurônio, visto que a resposta que se deseja é positiva ou negativa para um determinado fluxo. E a camada escondida possuirá uma quantidade de neurônios calculados a partir da média aritmética entre as camadas de entrada e saída.

        \item Treinamento do modelo: \\ O modelo proposto deve passar por um breve treinamento para que os erros sejam corrigidos antes que a rede comece a rebeber dados para serem analisados.

        \item Avaliar o modelo: \\ Uma vez com o modelo pronto é necessário avaliar a aplicabilidade do modelo em vista da acurácia obtida em relação a quantidade de reconhecimentos feitos em uma base de dados controlada e com um número conhecido de dados que representam fluxos maliciosos.

    \end{itemize}

\section{Evolução e Otimização do Modelo}
    De acordo com os resultados alcançados na fase anterior, o processo de evolução do modelo se dará no realocamento de neurônios nas três camadas, visando um melhor aprendizado, um melhor desempenho ou uma melhor precisão de resultados.






