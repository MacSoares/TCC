\chapter*[Introdução]{Introdução}
\addcontentsline{toc}{chapter}{Introdução}

Com o grande crescimento da conexão de computadores ao redor do mundo, é verificado um aumento significativo nos tipos e números de ataques a sistemas conectados em rede, o que gera uma complexidade maior para se planejar mecanismos de prevenção tradicionais. Baseado nesta crescente foram também bastante difundidos os sistemas de detecção de intrusão, que passaram a se tornar um componente importante de diversos sistemas de segurança.

Sendo assim, uma uma boa definição para um sistema de detecção de intrusão (Intrusion Detection Systems \textit{IDS}) é: uma ferramenta inteligente capaz de detectar tipos padronizados, por meio de regras, de tentativas de invasão que tem por objetivo a geração de alertas quando pacotes maliciosos são detectados. Entretanto este tipo de sistema tem o problema de que a cada vez que um novo tipo de ataque é feito é necessária a atualização das regras de comparação, isso gera um grande esforço para que sejam atualizadas essa base de regras e a base de dados, o que pode se tornar mais crítico com a demora para estas atualizações.

Este trabalho tem a proposta de através do uso de redes neurais artificiais aplicar um modelo de detecção que aprende a detectar tanto padrões conhecidos quanto novos padrões de  ataques efetuados.

Nas proximas sessões serão descritos os objetivos do trabalho e em seguida a fundamentação teórica e metodologia abordadas.