Os resultados alcançados no projeto se dividiram em duas fases diferentes: o \textit{software} GPSATWeb, desenvolvido para aplicação da ficha de avaliação dos amputados; a RNA, que apresenta uma análise da condição de pele dos amputados com a utilização fotos como dados de entrada.

O \textit{software web} está pronto e funcional. Esta aplicação visa facilitar e agilizar a aplicação e o preenchimento da ficha de avaliação dos pacientes amputados, além de salvar seguramente seus dados e gerar um relatório da avaliação contendo o histórico do paciente.

% A RNA é o primeiro passo para um trabalho futuro de uma aplicação inteligente capaz de automatizar parcialmente as avaliações feitas pelo profissional de saúde a fim de facilitar o diagnóstico de algumas partes da ficha de avaliação do paciente. No momento, ela apenas analisa imagens da pele dos cotos dos pacientes amputados para classificá-las, seguindo as denominações da ficha de avaliação.



- - - - falar sobre IA explicitamente - - - -  \\
- - - - descrever resultados de rna com estatística - - - -  