O conceito de desenvolvimento ágil nasceu em meio a uma necessidade global de desenvolvimento de softwares de maneira mais rápida e eficiente por parte das empresas, que precisavam entregar produtos que refletiam as necessidades dos clientes a um preço compeptitivo ao mercado. Com a crescente demanda de produtos e instabilidade dos requisitos exigidos pelos clientes, a metodologia ágil cresceu por sua flexibilidade para lidar com  mudanças nas necessidades e rápidas respostas a estas mudanças nas entregas de produtos \cite{Cheng1998}.

A partir desta crescente demanda mundial e para consolidar o método ágil de desenvolvimento, em 2001 foi criado então o Manifesto Ágil, o qual definia quatro valores para o desenvolvimento ágil de software que eram: 
	\begin{itemize}
		\item Indivíduos e interações entre eles mais que processos e ferramentas;
			\begin{itemize}
				\item As interações entre os indivíduos criam um ambiente produtivo melhor e motivador, logo com um ambiente de trabalho melhor as pessoas são mais produtivas.
			\end{itemize}
		\item Software em funcionamento mais que documentação abrangente;
			\begin{itemize}
				\item Sempre que se tem um produto com valor agregado ele pode ser enviado para aprovação do cliente, o que acelera o recebimento de receita pela empresa. Documentação não agrega tanto valor para o cliente e pode esperar.
			\end{itemize}
		\item Colaboração com o cliente mais que negociação de contratos;
			\begin{itemize}
				\item Um cliente que sente sua importância com o time de desenvolvimento nota a vontade de entrega de um produto melhor e colabora mais com a equipe.
			\end{itemize}
		\item Responder à mudanças mais que seguir um plano.
			\begin{itemize}
				\item O produto deve sofrer alterações de acordo com a necessidade do cliente, pois seu mundo muda constantemente. Sendo assim o software deve estar preparado para sofrer mudanças.
			\end{itemize}
	\end{itemize}

Desta maneira, a partir destes quatro valores, se consolidaram doze princípios para o desenvolvimento os quais eram encabeçados primordialmente pelo princípio que diz que a maior prioridade do desenvolvedor é satisfazer o cliente através da entrega contínua de software com valor agregado a cada nova entrega \cite{Manifesto2001}, para que assim pudessem estar mais próximos ao cliente e respondendo às suas mudanças de necessidades a cada nova entrega de software. O desenvolvimento ágil não é um processo prescrito, muito menos uma metodologia completa, pelo contrário, é um estilo de desenvolvimento que complementa os métodos existentes e quando usado adequadamente, tende a resultar em num desenvolvimento mais rápido e de qualidade \cite{Ambler2002}.


