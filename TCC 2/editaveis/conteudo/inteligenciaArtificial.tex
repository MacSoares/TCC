A inteligência artificial pode ser definida sob diversos pontos de vista os quais podem dizer respeito tanto à capacidade de pensamento quanto à capacidade raciocínio dos agentes inteligentes quando comparados a seres humanos no caso de pensamentos, e a sistemas ideais no caso do raciocínio \cite{norvig2014inteligencia}.
Sendo assim, pode-se entender a inteligência artificial como o ramo da computação onde se propõe criar dispositivos inteligentes capazes de simular uma atividade humana, sendo ela um pensamento, um raciocínio ou mesmo uma atitude.