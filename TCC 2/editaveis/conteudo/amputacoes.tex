O termo amputação deriva do latim com o significado de “em volta de” e “podar/retirar”, respectivamente, “ambi” e “putatio”, sendo assim pode-se definir amputação como uma retirada cirúrgica ou não, total ou parcial de um membro.  As principais causas de amputações são traumatismo, doenças vasculares periféricas, deficiência congênita, doenças infecciosas e patologias malignas \cite{Carvalho2003}, algumas menos frequentes como esmagamento e queimaduras térmicas e, ou elétricas (FRIEDMANN; 1994). As decisões das amputações devem ser tomadas com calma e precisão, afim do indivíduo ter tempo de amadurecer a respeito das modificações fisiológicas e se adaptarem psicologicamente (DOWNIE,1983).


As referências em relação às amputações possuem dados tão antigos que datam desde 1500 a.C. descritos em manuscrito indiano (\textit{Rig-Veda}) relatando a história da rainha Vishpla que teria tido o membro inferior amputado durante uma batalha (FERNANDES, 2007). Os relatos de amputações feitas de maneira cirúrgicas apenas foram realizadas no início da época pré-cristã, ainda que não houvesse características operatórias, apresentam-se como as primeiras cirurgias de amputação. Porém, não houve nenhum relato citando ou descrevendo as primeiras amputações transtibiais e transfemorias.


Outras evidências de amputações são pinturas em cavernas espanholas e francesas de aproximadamente 38 mil anos, em que apareciam mutilações de membros. Enquanto em um poema escrito em 3500 a.C., que relata a história de uma rainha de guerra, que teve um membro inferior amputado, apresenta a primeira referência de próteses, pois a rainha confeccionou uma prótese de ferro para retornar a guerra (PEDRINELLI, 2004; CARVALHO, 1999).

A incidência das amputações de membros inferiores no Brasil são de em média 40.000 amputações ao ano, tendo como principais causas complicações da diabetes e origens traumáticas, sendo que as causas são um dos fatores que influenciam na protetização e cicatrização \cite{Reis2012}.

\subsection{Amputação em membros inferiores}
    Segundo \cite{Barreto2013}, os membros inferiores possuem, via de regra, uma maior chance de serem submetidos à uma cirurgia de amputação em comparação aos membros superiores.

    As amputações de membros inferiores, de forma geral, apresentam níveis de amputação e diferentes observações para realização de processo cirúrgico.  O procedimento cirúrgico e os resíduos biológicos podem facilitar ou dificultar a adaptação do indivíduo e assim decidir o uso de prótese adequada (O‘SULLIVAN, 2005).

    A amputação transfemoral trata-se de uma retirada do membro com nível de corte entre a desarticulação do joelho e a articulação do quadril, com classificações de longa, média ou curta de acordo com o nível de preservação do comprimento do fêmur. Além disso, as amputações de membros inferiores causam alterações estruturais, mecânicas e metabólicas, sendo essas alterações formas de adaptações a nova condição corporal (SOUSA et al., 2017).

    Os resíduos biológicos, tais como ossos, irão sustentar os tecidos moles. Com isso, deverão ser seccionados de forma que distribuam as cargas para facilitar a protetização, não afetando os tecidos nobres próximos (PEDRINELLI, 2004). As articulações devem ser preservadas desde que sejam favoráveis a uma cicatrização com ausência de infecções parcial ou invasivas, afim de proporcionar uma reabilitação protética adequada e rápida, o que justifica a importância de uma boa avaliação para a prescrição da reabilitação (O‘SULLIVAN, 2005).

    Para a reabilitação protética, devem ser levado em concideração que as amputações transfemorais e transtibiais apresentam peculiaridades e diferentes níveis. No processo cirúrgico, deve-se ter o cuidado para que não haja saliências ou arestas ósseas e a musculatura posterior deve ser rebatida anteriormente para que ocorra formação de coxim (musculatura residual), afim de facilitar a mioplastia (sutura dos músculos e fáscias posteriores na fáscia profunda dos músculos anteriores, ou seja, fixação dos músculos antagonistas aos agonistas) e miodese (reinserção de musculatura ao ponto ósseo), pois tais procedimentos melhoram a propriocepção, circulação e controle de coto (CARVALHO, 1999). Além disso, vale ressaltar a importância de identificar e reparar as artérias e veias importantes para vascularização do coto.

    O coto é de extrema importância para a reabilitação do paciente, dor e desconforto, pois problemas como cicatrizes cutâneas aderentes ou invaginadas, deficiência nos tecidos e pele friável, são os maiores causadores de dor e desconforto. Porém, sabe-se que atualmente existem técnicas para corrigir e evitar tais problemas afim de reduzir as consequências negativas de uma amputação. Tais técnicas como meias em gel, diminuem tais impactos, bem como diminuem a dor e pressão sobre as zonas de impacto e torsão, reduzindo o gasto energético dos mesmos (CARVALHO, 1999; O‘SULLIVAN, 2005).

    As alterações a nível de coto residual e seus impactos no processo de reabilitação e protetização necessitam de uma avaliação fisioterapêutica e médica detalhada, para um bom norteamento para a condução do tratamento (PRIM et al., 2016). Sendo assim, esse trabalho permite a realização de forma automatizada de uma boa avaliação.
