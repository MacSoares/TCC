	O termo \textit{Representational State Transfer} (REST), define um conjunto de principios arquiteturais que podem ser usados para projetar serviços \textit{web} que trabalham com os recursos de algum sistema. Este tipo de arquitetura permite que se desenhe como os recursos de um sistema serão endereçados e transferido via requisições \textit{Hypertext Transfer Protocol}, ou Protocolo de Transferência de Hipertexto (HTTP) \cite{Rodriguez2008}. O HTTP é um protocolo baseado em documentos, no qual uma requisição é um envelope com um documento enviado a um servidor ou aplicação. Este documento normalmente pode conter qualquer informação, porém existem alguns pontos que devem ser explícitos, tais como o método, um endereço e os \textit{headres}, que são as palavras chaves da requisição \cite{Masse2011}.

    Desta maneira, atravez das requisições HTTP, o sistema \textit{web} se comunica com a \textit{api} utilizando os métodos padrões HTTP \textit{post}, \textit{delete}, \textit{get} e \textit{put} \cite{Rodriguez2008}, os quais qualificam a ação a ser desencolvida sobre os dados pela \textit{api}. O método \textit{post} faz a \textit{api} receber novos dados e armazenar na base de dados, já o \textit{delete} faz com que ela delete os dados segundo um identificador do dado a ser deletado. Já os métodos \textit{get} e \textit{put} são utilizados para envio de dados ou atualização de dados respectivamente, visto que o método \textit{put} também pode ser utilizado para armazenar novos dados, mas por um padrão deste projeto está sendo utilizado somente para atualização de dados.

   
    
