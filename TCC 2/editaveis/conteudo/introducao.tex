\part{Introdução}
\chapter[Introdução]{Introdução}
% \addcontentsline{toc}{chapter}{Introdução}

Com o grande crescimento da conexão de computadores ao redor do mundo, é verificado um aumento significativo nos tipos e números de ataques a sistemas conectados em rede, o que gera uma complexidade maior para se planejar mecanismos de prevenção tradicionais. Baseado nesta crescente foram também bastante difundidos os sistemas de detecção de intrusão, que passaram a se tornar um componente importante de diversos sistemas de segurança.

Neste trabalho será apresentada uma proposta de modelagem utilizando conceitos de aprendizado de máquina a fim de atacar o problema de detecção de ataques em redes de computadores. Esta proposta visa investigar o uso de redes neurais artificiais, do modelo MLP e utilizando o algorítmo de treinamento de \textit{Backpropagation}, para reconhecer padrões já conhecidos de ataque e se possível detectar novos padrões de ataques efetuados.

Estes sistemas conectados trabalham para que os dados sejam colhidos de maneira mais especifica dentro de um grande volume de dados existentes e que podem estar ou não integrados com outros tipos de dados. Sendo assim esta proposta se encontra dentro de um Big Data, que segundo \cite{Dumbill2012} são dados que excedem a capacidade de processamento e não se encaixam nas estruturas de bancos de dados convencionais.

Desta maneira os conceitos de aprendizado de máquina serão utilizados como uma abordagem para auxiliar na extração de dados destes modelos de Big Data. Isso pois dispõe de métodos de seleção de características capazes de percorrer esa grande massa de dados e extrair com determinada exatidão cada dado descrito segundo as características determinadas.

Nas proximas sessões serão descritos os objetivos do trabalho e em seguida a fundamentação teórica e metodologia abordadas.