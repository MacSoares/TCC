\part{Introdução}
\chapter[Introdução]{Introdução}
% \addcontentsline{toc}{chapter}{Introdução}

No Brasil de hoje, é crescente a quantidade de pacientes amputados. Existem estimativas de que exitam aproximadamente 40.000 amputações por ano em solo brasileiro \cite{Reis2011}. Baseado nesta demanda, é necessário um meio de melhor servir estes tipos de pacientes que necessitam de um atendimento mais detalhado e consistente.

Neste trabalho é apresentada uma proposta de avaliação destes pacientes,  utilizando além de conceitos de aprendizado de máquina, uma ficha de avaliação dos pacientes baseada na Classificação Internacional de Funcionalidade, Incapacidade e Saúde (CIF) \cite{OMS2004}, redigida pela Organização Mundial de Saúde (OMS), que apresenta uma opção de melhor avaliação dos pacientes amputados.

Esta proposta investiga o uso de RNA, do modelo MLP e utilizando o algoritmo de treinamento conhecido como \textit{Backpropagation}, para auxílio no reconhecimento de padrões de pele já conhecidos dos cotos dos pacientes e se possível classificá-los com uma boa precisão. Os conceitos de aprendizado de máquina são utilizados, neste contexto, como uma abordagem para auxiliar na precisão dos diagnósticos.

Nas próximas sessões são descritos os objetivos do trabalho e, em seguida, a fundamentação teórica e metodologia abordadas.