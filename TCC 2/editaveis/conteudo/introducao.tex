\part{Introdução}
\chapter[Introdução]{Introdução}
% \addcontentsline{toc}{chapter}{Introdução}

No Brasil de hoje, os atendimentos prestados a pacientes amputados transfemurais e transtibiais nem sempre é o mais adequado ou completo, visto que muitos acabam por não conseguirem proteses adequadas. Baseado nesta demanda, vê-se necessário um meio de melhor servir estes tipos de pacientes que necessitam de um atendimento mais detalhado e consistente.

Neste trabalho será apresentada uma proposta de avaliação destes pacientes  utilizando além de conceitos de aprendizado de máquina, uma ficha de avaliação dos pacientes baseada na Classificação Internacional de Funcionalidade, Incapacidade e Saúde (CIF), redigida pela Organização Mundial de Saúde (OMS), a fim de atacar o problema de má avaliação dos pacientes amputados. Esta proposta visa investigar o uso de redes neurais artificiais, do modelo MLP e utilizando o algorítmo de treinamento de \textit{Backpropagation}, para auxílio no reconhecimento de padrões de pele já conhecidos dos cotos dos pacientes e se possível classificá-los com uma boa precisão. Desta maneira os conceitos de aprendizado de máquina serão utilizados como uma abordagem para auxiliar na precisão dos diagnósticos.

Nas proximas sessões serão descritos os objetivos do trabalho e em seguida a fundamentação teórica e metodologia abordadas.