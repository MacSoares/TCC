\part{Introdução}
\chapter[Introdução]{Introdução}
% \addcontentsline{toc}{chapter}{Introdução}

Atualmente, é crescente a quantidade de pacientes amputados. Estima-se que exitam aproximadamente 40.000 amputações por ano em solo brasileiro \cite{Reis2012}. Baseado nesta demanda, é necessário um meio de melhor servir estes tipos de pacientes que necessitam de um atendimento mais detalhado e consistente.

Neste trabalho é apresentada uma proposta de avaliação destes pacientes,  utilizando além de conceitos de aprendizado de máquina, uma ficha de avaliação dos pacientes baseada em determinados módulos da Classificação Internacional de Funcionalidade, Incapacidade e Saúde (CIF) \cite{OMS2004}, redigida pela Organização Mundial de Saúde (OMS), que apresenta uma opção de melhor avaliação dos pacientes amputados de membros inferiores.

Esta proposta investiga o uso de RNA, do modelo MLP utilizando o algoritmo de treinamento supervisionado conhecido como \textit{Backpropagation}, para auxílio no reconhecimento de padrões de pele já conhecidos dos cotos dos pacientes e, se possível, classificá-los com uma boa precisão para assim, auxiliar o profissional de saúde na precisão de sua tomada de deçisão na classificação do tipo de pele do coto do paciente amputado de membro inferior. Os conceitos de aprendizado de máquina são utilizados, neste contexto, como uma abordagem para auxiliar na precisão dos diagnósticos.

Nas próximas sessões são descritos os objetivos do trabalho e, em seguida, a fundamentação teórica e metodologia abordadas, além dos resultados alcançados, passos futuros e conciderações finais.