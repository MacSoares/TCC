Os objetivos iniciais do trabalho foram atingidos com a construção do sistema \textit{web} e da RNA que reconhece e classifica diferentes tipos de pele, tal como proposto.

Como passos futuros, é necessária a integração destes dois produtos que, hoje, funcionam de forma separada e necessitam de que um profissional de computação faça o tratamento dos dados de imagem a serem passados de um para outro. Também será necessária uma evolução no \textit{software} GPSATWeb para que o mesmo faça o tratamento necessário nas imagens para envio à RNA.  

Outra necessidade futura será a evolução da RNA para que a mesma identifique e classifique automaticamente mais dados descritos na ficha de avaliação do paciente para ajudar o profissional de saúde em seu diagnóstico. Isso pode ser feito, por exemplo automatizando o processo de parametrização do algorítmo de ML utilizando-se da ferramenta \textit{GridSearch}, presente no já utilizado \textit{Scikit Learn}. Outra forma de melhorar o desempenho da RNA é testar composições de imagens para definir um tipo de pele utilizando imagnes de diferentes ângulos do coto. Além disso, é necessário também um maior número de coletas de dados para que o treinamento e a validação de resultados da RNA possam ser cada vez melhores e mais precisos. 

A taxa de acerto relativamente baixa da RNA MLP construída se deve à reduzida amostra de dados em que ela foi trabalhada. Isso pois, o número reduzido de pacientes voluntários que participaram do trabalho não foi capaz de gerar uma massa de dados satisfatoriamente grande, por ser uma população de difícil acesso. Mesmo assim, pode-se afirmar que o sistema desenvolvido pode ser utilizado no campo da saúde para auxílio ao profissional que faz os exames nos pacientes ou para treinamento do mesmo visando a melhoria e o aumento da velocidade no atendimento ao paciente amputado.

Por fim, conclui-se que a aplicação e evolução posterior destes dois sistemas em conjunto pode ajudar cada vez mais os profissionais de saúde a serem mais rápidos e efetivos na avaliação dos pacientes amputados de membro inferior, tanto em um aspecto geral, visto que este sistema foi contrsuído exclusivamente para este tipo de avaliação, quanto na avaliação específica do tipo de pele dos cotos dos pacientes, auxiliado nesse aspecto pela resposta dada pela RNA.