
Sistemas de detecção de intrusão, ou Intrusion Detection Systems (\textit{IDS}), são sistemas capazes de detectar atividades suspeitas em pacotes de dados a partir da análise de regras de uma base de conhecimento. Existem dois tipos bastante difundidos de formas de um \textit{IDS} se comportar para a detecção de malwares, o primeiro baseado em detecção por assinatura e o segundo baseado em detecção por anomalias.

\begin{itemize}
    \item O \textit{IDS} baseado em detecção por assinatura funcionam da seguinte maneira: o software busca em sua base de dados de assinaturas suspeitas, pelas assinaturas recebidas nos pacotes de dados, caso ele encontre, o pacote é marcado como malicioso. \cite{rehman2003intrusion}
    \item O \textit{IDS} baseado em detecção por anomalia funciona da seguinte maneira: a partir da leitura de muitos fluxos de pacotes, o sistema obtem um padrão de comportamento das atividades, no momento em que um pacote se comporta de maneira diferente o \textit{IDS} capta essa diferença e a trata como anomalia, marcando o pacote. \cite{debar2000introduction}
\end{itemize}


Entretanto este tipo de sistema baseado em assinaturas tem o problema de que a cada vez que um novo tipo de ataque é feito é necessária a atualização das regras de comparação, isso gera um grande esforço para que sejam atualizadas essa base de regras e a base de dados, o que pode se tornar mais crítico com a demora para estas atualizações. Outro problema associado a \textit{IDS}'s é que normalmente eles somente emitem sinais de que um pacode pode conter uma atividade maliciosa, e não chega a categorizá-la.

Os \textit{IDS} são categorizados de acordo com a fonte de dados principal da qual consomem. Sendo assim, as duas categorias mais difundidas são:
\begin{itemize}
    \item \textit{Network IDS} - Também conhecidos como NIDS, estes sistemas de detecção são ligados na rede que se deseja monitorar, seja ela cabeada ou wifi. Estes detectores capturam pacotes de dados que transitam pela rede para sua análise e a partir daí pode assumir, normalmente o comportamento de detecção por assinatura.
    \item \textit{Host IDS} -  Também conhecidos como HIDS, estes sistemas de detecção estão ligados diretamente a um servidor, podendo fazer suas análises somente nesse servidor sem poder fazer acesso a subrede de dados. Estes detectores, normalmente se comportam detectando intrusões por anomalias encontradas nos fluxos de dados de arquivos e logs do servidor. \cite{rehman2003intrusion}
\end{itemize}