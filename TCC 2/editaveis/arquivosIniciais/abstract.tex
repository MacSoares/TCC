\begin{resumo}[Abstract]
 \begin{otherlanguage*}{english}

   There is a growing number of people suffering from lower limb amputation in Brazil, and many of these people are fatally not properly cared by health professionals, resulting in uncertain evaluations and diagnoses. This paper presents a proposal for the evaluation of transfemoral and transtibial amputees using an assessment sheet based on the International Classification of Functioning (CIF) published by the World Health Organization (WHO) and in the use of artificials neural networks (ANN) build in the Multi Layer Perceptron model (MLP) and supervised training. The objective is to have an intelligent system capable of presenting a report on the diagnoses. For this, data were collected with 13 patients generating 195 archives of images that served as input to the ANN built. The highest reached by MLP ANN was 77.54\%. A software has also been build as an application that make saves and subsequent retrieval of data from the evaluation sheets, who could make the evaluation process accelerated and safer.

   \vspace{\onelineskip}

   \noindent
   \textbf{Key-words}: Multi Layer Perceptron. Backpropagation. Machine Learning. CIF. WHO. Transfemoral. Transtibial.
 \end{otherlanguage*}
\end{resumo}
