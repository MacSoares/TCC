\begin{resumo}

 É crescente o número de pessoas que sofrem amputação de membros inferiores no Brasil, e muitas destas pessoas fatalmente não são atendidas adequadamente por profissionais da saúde, resultando em avaliações e diagnósticos incertos. Este trabalho vem apresentar uma proposta de avaliação de pacientes amputados transfemorais e transtibiais fazendo uso de uma ficha de avaliação baseada na Classificação Internacional de Funcionalidade (CIF) publicada pela Organização Mundial de Saúde (OMS) e no uso de Redes Neurais Artificiais (RNA) do modelo \textit{Multi Layer Perceptron} (MLP) e treinamento supervisionado. O objetivo é possuir um sistema inteligente capaz de apresentar um relatório sobre a avaliação efetuada. Para isso foram feitas coletas de dados com 13 pacientes amputados de membro inferior gerando 195 arquivos de imagens que serviram de entrada para a RNA construída. A maior taxa de acerto atingida pela RNA MLP foi de 77.54\%. Também foi construído e testado durante uma coleta de dados, um \textit{software} para que a aplicação e posterior recuperação de dados das fichas de avaliação pudessem ser acelerados e mais seguros.



 \vspace{\onelineskip}

 \noindent
 \textbf{Palavras-chaves}: Multi Layer Perceptron. Backpropagation. Aprendizado de Máquina. CIF. OMS. Transfemorais. Transtibiais.
\end{resumo}
